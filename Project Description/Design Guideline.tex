\documentclass{article}
\usepackage{graphicx} % Required for inserting images
\usepackage{hyperref}

\title{Design Guideline}
\author{Authors: }
\date{Somethingary 2025}

\begin{document}

\maketitle

\newpage
\section{Concept phase}
In the concept phase, before writing code, you should decide the general rules and direction:
\begin{enumerate}
    \item Theme: The design should be serious, since it's a car rental rental company, which should show professionalism.
    \item Color scheme:
    \begin{itemize}
        \item The color theme of the company is Black, Orange and White.
         \item main-color: #1A1A1A;
                  \item secondary-color: #FF5F00;
                  \item background-color: #FFFFFF;
    \end{itemize}
    \item Hierarchy and layout:
    \begin{itemize}
        \item What is the most important message/product on the page?
        \begin{itemize}
            \item The most important product are the cars that are being displayed.
        \end{itemize}
        \item What is the order of importance for the different elements?
        \begin{enumerate}
            \item Availability
            \item Price
            \item Type
            \item Other details.
        \end{enumerate}
        \item What sections will there be on the page?
        \begin{itemize}
            \item Rentalpage
            \item Bookingpage
            \item Homepage.
            \item Accountpage
        \end{itemize}
        \item What layout(s) will you use for the sections?
        \begin{itemize}
            \item We have a nav-bar in the header. Which consist of logo, time-picker and login button.
            \item In the hero section we have the date-time-picker for pickup and dropoff with a 'show cars' call to action button.
            \item The booking page's hero section consists of the booking details
            \item In the footer we have contacts and social network.
            \item The rental page's hero section consists of all available cars.
        \end{itemize}
    \end{itemize}
    \item Images:
    \begin{itemize}
        \item Will there be text on images or aside images?
        \begin{itemize}
            \item There will be text on the side of the images.
        \end{itemize}
        
        There will be information on the images. 
        \item Will we use clear images or a color-overlay?
        \begin{itemize}
            \item Depends on what image it is, and where it will be used.
        \end{itemize}
        \item Will there be blur for images?
        \begin{itemize}
            \item Depends on what image it is, and where it will be used.
        \end{itemize}
    \end{itemize}
    \item Icons:
    \begin{itemize}
        \item Will we use icons and for what?
        \begin{itemize}
            \item Yes, to indicate certain actions, which are mostly universal in design, to improve user interaction.
        \end{itemize}
        \item Colored or black-and-white/greyscale icons?
        \begin{itemize}
            \item Not colored, only black/white/greyscaled.
        \end{itemize}
    \end{itemize}
    \item Typography:
    \begin{itemize}
        \item Should it by a Serif or Sans-Serif font?
        \begin{itemize}
            \item Sans-Serif
        \end{itemize}
        \item Bold or thin font?
        \begin{itemize}
            \item Thin on some, bold on some. It depends on the importance of each text-section.
        \end{itemize}
        \item Rounded or sharp edges?
        \begin{itemize}
            \item We use mostly sharp edges.
        \end{itemize}
    \end{itemize}
    \item Border rounding:
    \begin{itemize}
        \item How rounded should be the edges, according to the theme?
        \begin{itemize}
            \item 15 degrees
        \end{itemize}
    \end{itemize}
        \item Shadows:
    \begin{itemize}
        \item How much shadow will we use in general, according to the theme?
        \begin{itemize}
            \item We use shadows in most elements, according to theme.
        \end{itemize}
    \end{itemize}
\end{enumerate}

\section{Sketching}
Create a sketch of the page which shows the general layout of the page.
\href{https://www.figma.com/design/Bad0CJ6dLRK9TCG3znJ58A/Rental-Roulette?node-id=448-2&p=f&t=oREOp7yr8ViVVVtf-0}{This is the figma.}


\section{Specific Technical Decisions }
This comes at a later stage when you start implementing the site:
\begin{enumerate}
    \item Colours:
    \begin{itemize}
        \item 1A1A1A - Black
        \item FF5F00 - Orange
        \item FFFFFF - White
        \item We generate different tints based on color strength.
    \end{itemize}
    \item Typography:
    \begin{enumerate}
        \item[a] Choose a font for titles (You will probably want to change it when you see the site, but
start with something) - Poppins, Serif.
        \item[b] If body text needs another font, choose it. Preferably, use the same font for all texts.
        We use Poppins
        \item[c] Define font scale: based on 14px
        \begin{enumerate}
            \item           --font-size-xxxl: 2.5rem; 
         \item --font-size-xxl: 1.8rem;
                 \item  --font-size-xl: 1.5rem;
                 \item --font-size-l: 1rem;
                 \item --font-size-m: 0.85rem;
                \item  --font-size-s: 0.75rem;
        \end{enumerate}

    \end{enumerate}
    \item Hierarchy and layout:
    \begin{enumerate}
        \item[a] Which elements will need an accent because we want to increase their priority in the
hierarchy?
        \item[b] What components will we need to implement the necessary layouts?
    \end{enumerate}
    \item Icons:
    \begin{enumerate}
        \item[a] Choose icon pack if you need one: \href{https://phosphoricons.com/}{Icon samples}
    \end{enumerate}
    \item  Spacing:
        \begin{enumerate}
        \item[a] Define spacing scale, f.ex, multiples of 16px (multiples of 4px for smaller padding)
    \end{enumerate}
    \item  Border rounding:
        \begin{enumerate}
        \item[a] Synchronize the chosen font with roundedness (very rounded font → more border
rounding for elements)
        \item[b]  Border rounding amount for general elements: cards, forms, images, etc
        \item[c] Border rounding for buttons
    \end{enumerate}
\end{enumerate}

\end{document}